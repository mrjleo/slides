\documentclass{standalone}
% includes
\usepackage{pgfplots}
\usepackage{tikz}
\usepackage{tikz-3dplot}
\usepackage{xcolor}
\usepackage{soul}
\usepackage{siunitx}
\usepackage{marvosym}
\usepackage{amsmath}
\usepackage{amssymb}
\usepackage{mathtools}
\usepackage{tabularx}
\usepackage{booktabs}
\usepackage[
    most,
]{tcolorbox}

% fonts
\usepackage{libertine}
\usepackage{inconsolata}
\usepackage{fontsize}

\changefontsize{14pt}

% layout
\newcommand{\hide}[1]{\textcolor{gray}{#1}}

% allow bolding numbers in \num{}
\sisetup{
    text-series-to-math=true,
    propagate-math-font=true,
}

% reveal theme background (for transparency hack)
\definecolor{background}{HTML}{F0F1EB}

% colors
\colorlet{plotColorNeutral}{gray}
\definecolor{plotColor1}{HTML}{e41a1c}
\definecolor{plotColor2}{HTML}{377eb8}
\definecolor{plotColor3}{HTML}{4daf4a}
\definecolor{plotColor4}{HTML}{984ea3}
\colorlet{plotColorNeutral*}{plotColorNeutral!40}
\colorlet{plotColor1*}{plotColor1!40}
\colorlet{plotColor2*}{plotColor2!40}
\colorlet{plotColor3*}{plotColor3!40}
\colorlet{plotColor4*}{plotColor4!40}
\pgfplotsset{
    colormap={greenred}{HTML=(4daf4a) HTML=(e41a1c)},
    colormap={redgreen}{HTML=(e41a1c) HTML=(4daf4a)}
}

\definecolor{hlColorRed}{HTML}{DB4437}
\definecolor{hlColorGreen}{HTML}{0F9D58}
\definecolor{hlColorBlue}{HTML}{4285F4}
\newcommand{\textRed}[1]{\textcolor{hlColorRed}{#1}}
\newcommand{\textGreen}[1]{\textcolor{hlColorGreen}{#1}}
\newcommand{\textBlue}[1]{\textcolor{hlColorBlue}{#1}}
\newcommand{\texthighlight}[1]{%
    \begingroup%
    \sethlcolor{plotColor3*}%
    \hl{#1}%
    \endgroup%
}
\newtcbox{\hldashed}[1][]{
    size=fbox,
    boxsep=2pt,
    boxrule=0pt,
    enhanced,
    borderline={0.5pt}{0pt}{dashed},
    on line,
    #1
}
\newtcbox{\hldotted}[1][]{
    size=fbox,
    boxsep=2pt,
    boxrule=0pt,
    enhanced,
    borderline={0.5pt}{0pt}{dotted},
    on line,
    #1
}
\newcommand{\midrulesep}{
    \arrayrulecolor{gray}
    \midrule[0.25pt]
    \arrayrulecolor{black}
}
\newcommand{\tablearrow}{{} \rotatebox[origin=c]{180}{$\Lsh$} {}}
\newcommand{\sigdef}[1]{{\footnotesize \texttt{[#1]}}}
\newcommand{\sigimpr}[1]{\footnotesize \textsuperscript{\texttt{[#1]}}}

% pgf, tikz
\pgfplotsset{compat=1.15}
\usepgfplotslibrary{
    statistics,
    colorbrewer,
    groupplots,
}
\usetikzlibrary{
    patterns,
    shapes.geometric,
    decorations.text,
    matrix,
    fit,
    backgrounds,
    positioning,
}

% required for patterns in boxplots
\makeatletter
\tikzset{nomorepostaction/.code=\let\tikz@postactions\pgfutil@empty}
\makeatother

% tikz styles
\tikzset{
    fignode/.style={
            outer sep=0.25em,
        }
}

\tikzset{
    framedfignode/.style={
            outer sep=0.25em,
            inner sep=0.5em,
            rounded corners,
            fill=white,
            draw,
        }
}

% LLM
\newcommand{\bert}{\textsc{BERT}}
\newcommand{\bertbase}{$\text{BERT}_\text{base}$}
\newcommand{\wordpiece}{\textsc{WordPiece}}
\newcommand{\adam}{\textsc{Adam}}
\newcommand{\adamw}{\textsc{AdamW}}

% embeddings
\newcommand{\wordtovec}{\textsc{Word2Vec}}
\newcommand{\glove}{\textsc{GloVe}}
\newcommand{\sbert}{\textsc{Sentence-BERT}}

% ranking
\newcommand{\ql}{\textsc{QL}}
\newcommand{\dssm}{\textsc{DSSM}}
\newcommand{\desm}{\textsc{DESM}}
\newcommand{\doctquery}{\textsc{docT5query}}
\newcommand{\monobert}{\textsc{monoBERT}}
\newcommand{\dpr}{\textsc{DPR}}
\newcommand{\ance}{\textsc{ANCE}}
\newcommand{\colbert}{\textsc{ColBERT}}
\newcommand{\colberter}{\textsc{ColBERTer}}
\newcommand{\tct}{\textsc{TCT-ColBERT}}
\newcommand{\tildemodel}{\textsc{TILDE}}
\newcommand{\tildeone}{\textsc{TILDEv1}}
\newcommand{\tildetwo}{\textsc{TILDEv2}}
\newcommand{\splade}{\textsc{SPLADE}}
\newcommand{\led}{\textsc{LED}}
\newcommand{\deepimpact}{\textsc{DeepImpact}}
\newcommand{\clear}{\textsc{CLEAR}}
\newcommand{\coil}{\textsc{COIL}}
\newcommand{\unicoil}{\textsc{uniCOIL}}
\newcommand{\coilcr}{\textsc{COILcr}}
\newcommand{\bm}{\textsc{BM25}}
\newcommand{\bmp}{\textsc{BM25P}}
\newcommand{\rmprf}{\textsc{RM3}}
\newcommand{\deepct}{\textsc{DEEP-CT}}
\newcommand{\bertcls}{\textsc{BERT-CLS}}
\newcommand{\berts}{\textsc{BERT-3S}}
\newcommand{\doclabeled}{\textsc{Doc-Labeled}}
\newcommand{\matchpyramid}{\textsc{MatchPyramid}}
\newcommand{\pacrr}{\textsc{PACRR}}
\newcommand{\copacrr}{\textsc{Co-PACRR}}
\newcommand{\convknrm}{\textsc{Conv-KNRM}}
\newcommand{\duet}{\textsc{DUET}}
\newcommand{\tkl}{\textsc{TKL}}
\newcommand{\tklsmall}{\textsc{TKL-2k}}
\newcommand{\ranknet}{\textsc{RankNet}}
\newcommand{\knrm}{\textsc{K-NRM}}
\newcommand{\dmn}{\textsc{DMN}}
\newcommand{\qalstm}{\textsc{QA-LSTM}}

% metrics
\newcommand{\ap}{AP}
\newcommand{\rr}{RR}
\newcommand{\recall}{R}
\newcommand{\precision}{P}
\newcommand{\dcg}{DCG}
\newcommand{\idcg}{IDCG}
\newcommand{\ndcg}{nDCG}

% efficient BERT
\newcommand{\powerbert}{\textsc{PoWER-BERT}}
\newcommand{\skipbert}{\textsc{SkipBERT}}
\newcommand{\deebert}{\textsc{DeeBERT}}

% datasets
\newcommand{\msmarco}{\textsc{MS MARCO}}
\newcommand{\msmpsgdev}{\textsc{MSM-Psg-Dev}}
\newcommand{\trecdlpsgn}{\textsc{TREC-DL-Psg'19}}
\newcommand{\trecdlpsgt}{\textsc{TREC-DL-Psg'20}}
\newcommand{\msmdocdev}{\textsc{MSM-Doc-Dev}}
\newcommand{\trecdldocn}{\textsc{TREC-DL-Doc'19}}
\newcommand{\trecdldoct}{\textsc{TREC-DL-Doc'20}}
\newcommand{\beir}{\textsc{BEIR}}
\newcommand{\beirmsm}{\textsc{MS MARCO}}
\newcommand{\beirfever}{\textsc{Fever}}
\newcommand{\beircfever}{\textsc{Climate-Fever}}
\newcommand{\beirfiqa}{\textsc{FiQA}}
\newcommand{\beirquora}{\textsc{Quora}}
\newcommand{\beirnq}{\textsc{NQ}}
\newcommand{\beirhpqa}{\textsc{HotpotQA}}
\newcommand{\beirdbp}{\textsc{DBpedia-Entity}}
\newcommand{\beirscifact}{\textsc{SciFact}}
\newcommand{\beirscidocs}{\textsc{SciDocs}}
\newcommand{\beirtc}{\textsc{TREC-Covid}}
\newcommand{\beirtouche}{\textsc{Webis-Touché-2020}}
\newcommand{\beirnfc}{\textsc{NFCorpus}}
\newcommand{\core}{\textsc{Core17}}
\newcommand{\clueweb}{\textsc{ClueWeb09}}
\newcommand{\antique}{\textsc{ANTIQUE}}
\newcommand{\insrqa}{\textsc{InsuranceQA}}

% Fast-Forward
\newcommand{\fastforward}{\textsc{Fast-Forward}}
\newcommand{\sparseretrieval}{\textsc{Sparse Retrieval}}
\newcommand{\denseretrieval}{\textsc{Dense Retrieval}}
\newcommand{\hybrid}{\textsc{Hybrid Retrieval}}
\newcommand{\reranking}{\textsc{Re-Ranking}}
\newcommand{\interpolatedreranking}{\textsc{Interpolation}}
\newcommand{\pyserini}{\textsc{Pyserini}}
\newcommand{\faiss}{\textsc{FAISS}}
\newcommand{\selbert}{\textsc{Selective-BERT}}

% BERT-DMN
\newcommand{\blite}{$\text{BERT}_\text{lite}$}
\newcommand{\bdmn}{$\text{BERT-DMN}$}
\newcommand{\bdlite}{$\text{BERT-DMN}_\text{lite}$}

% S&R
\newcommand{\sr}{\textsc{Select-And-Rank}}
\newcommand{\sratt}{\textsc{S\&R-ATT}}
\newcommand{\srlin}{\textsc{S\&R-LIN}}
\newcommand{\plrnd}{\textsc{PL-RND}}
\newcommand{\plbert}{\textsc{PL-BERT}}
\newcommand{\pllstm}{\textsc{PL-LSTM}}
\newcommand{\plbm}{\textsc{PL-BM25}}
\newcommand{\plsem}{\textsc{PL-SEM}}

\newcommand{\otree}{\textsc{oTree}}

% BoilerNet
\newcommand{\boilernet}{\textsc{BoilerNet}}
\newcommand{\wtt}{\textsc{Web2Text}}
\newcommand{\bp}{\textsc{BoilerPipe}}
\newcommand{\moz}{\textsc{Readability.js}}
\newcommand{\bte}{\textsc{BTE}}
\newcommand{\cetd}{\textsc{CETD}}

\newcommand{\cleaneval}{\textsc{CleanEval}}
\newcommand{\gtrends}{\textsc{GoogleTrends-2017}}


\begin{document}
\begin{tabularx}{27cm}{crl}
    \toprule
    \multicolumn{2}{r}{\textbf{Rank}} & \textbf{Sentence}                                                                                                                               \\
    \midrule
    \multicolumn{3}{l}{\textbf{Document}: \texttt{Commodore\_(rank)}, \textbf{Rank}: \textGreen{\textbf{\#\num{3}}}}                                                                    \\
    \cmidrule{1-2}
    \parbox[b]{0pt}{\multirow{5}{*}{\rotatebox[origin=c]{90}{selected}}}
                                      & \bfseries \num{1} & A commodore's ship is typically designated by the flying of a Broad pennant, as opposed to an admiral's flag.               \\
                                      & \bfseries \num{2} & \texthighlight{Commodore is a naval rank used in many navies that is superior to a navy captain, but below a rear admiral.} \\
                                      & \bfseries \num{3} & It is sometimes abbreviated: as "Cdre" in British Royal Navy, "CDRE" in the US Navy [...]                                   \\
                                      & \bfseries \num{4} & Commodore (rank).                                                                                                           \\
                                      & \bfseries \num{5} & Non-English-speaking nations often use the rank of flotilla admiral or counter admiral [...]                                \\
    \cmidrule{1-2}
                                      & \num{6}           & As an official rank, a commodore typically commands a flotilla or squadron of ships [...]                                   \\
                                      & \num{7}           & Traditionally, "commodore" is the title for any officer assigned to command more than one ship [...]                        \\
                                      & \num{8}           & It is often regarded as a one-star rank with a NATO code of OF-6 [...]                                                      \\
    \midrule
    \multicolumn{3}{l}{\textbf{Document}: \texttt{Rear\_admiral}, \textbf{Rank}: \textRed{\textbf{\#\num{23}}}}                                                                         \\
    \cmidrule{1-2}
    \parbox[b]{0pt}{\multirow{5}{*}{\rotatebox[origin=c]{90}{selected}}}
                                      & \bfseries \num{1} & In the German Navy the rank is known as Konteradmiral, superior to the flotilla admiral (Commodore in other navies).        \\
                                      & \bfseries \num{2} & In the Royal Netherlands Navy, this rank is known as schout-bij-nacht (lit.                                                 \\
                                      & \bfseries \num{3} & [...] and in the Canadian Forces' French rank translations, the rank of rear admiral is known as contre-amiral.             \\
                                      & \bfseries \num{4} & In some European navies (e.g.                                                                                               \\
                                      & \bfseries \num{5} & In many navies it is referred to as a two-star rank (OF-7).                                                                 \\
    \cmidrule{1-2}
                                      & ...               & ...                                                                                                                         \\
                                      & \num{13}          & \texthighlight{Rear admiral is a naval commissioned officer rank above that of a commodore and captain, and below [...]}    \\
                                      & \num{14}          & Each naval squadron would be assigned an admiral as its head, who would command from the centre vessel [...]                \\
    \midrule
    \multicolumn{3}{l}{\textbf{Query}: ``\textit{Commodore is ranked above a rear admiral.}'' (query ID 204575)}                                                                        \\
    \bottomrule
\end{tabularx}
\end{document}
